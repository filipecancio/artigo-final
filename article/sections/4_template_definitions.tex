\section{Conceitos de template}~\label{sec:template}

Exite dois conceitos de template apresentados no artigo: O ``template LaTex'' é referente ao conjunto de configurações em Latex e o ``template de repositório'' que corresponde a todo projeto apresentado no artigo. O template de repositório do nosso projeto contém o template da SBC como template LaTex.

\subsection{Template LaTex}
Para a elaborar um artigo precisamos seguir um conjunto de regras e normas. Um bom exemplo delas é a norma da ABNT - Assossiação Brasileira de Normas Técnicas. Com base nela, cada universidade e organização pode criar seu próprio modelo de artigo repetindo regras e formatos, ou melhorando eles. Com a linguagem LaTex podemos definir esse conjuntos de regras no que chamamos de template. Esse template possui diversos arquivos com extensões ``.bib, .bst, .sty'', entre outros que carregam consigo o formato das letras, fonte, cores, margem que formatam automaticamente o texto nas normas expecificadas. Os únicos arquivos necessários para criar os artigos acadêmicos são os arquivos de texto nomeados no formato ``.tex'' e quando for adicionar novas referências bibliográficas, um único arquivo no formato ``.bib''

\subsection{Template de repositório}
Cada projeto na plataforma do github é chamada de repositório, os repositórios são como pastas onde você guardar e versionar o código fonte presente no projeto. No Github é possível transformar um repositório em um template, assim ao criar novos repositórios, usando um repositório template, você cria um novo projeto, idêntico ao template escolhido.


Com os dois conceitos acima podemos entender que nesse projeto chamado ``sbc-template'', podemos criar novos projetos semelhantes e usar as mesmas configurações pra criar projetos com outras normas como ABNT por exemplo.