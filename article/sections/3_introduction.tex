\section{Introdução}

Existem diversas plataformas para digitar um texto acadêmico e entre elas temos a LaTex que na prática é uma linguagem de marcação, como HTML, onde permite ao usuário escrever textos acadêmicos se atendo à sintaxe. Não é tão simples configurar um projeto com LaTex, na verdade requer um conhecimento intermediário na linguagem para criar um template. Algumas entidades como a SPC - Sociedade Brasileira de Computação fornecem um template pronto que facilita a utilização, porém ainda é necessário tempo e conhecimento para configurar uma ferramenta que processe e compile o LaTex para o formato pdf.
A necessidade de conhecimento prévio na configuração de um ambiente, pode afastar acadêmicos das diversas áreas, da utilização do formato LaTex como ferramenta de elaboração de textos acadêmicos.
O artigo presente aborda a elaboração de um template de repositório do github que facilita o uso e desenvolvimento de artigos no formato Latex permitindo o usuário não precisar configurar um processador LaTex (para uso online), ou com uma configuração mínima (para uso offline).
O projeto utiliza o tempate SPC para a elaboração do projeto no GitHub.